\documentclass[../main.tex]{subfiles}
\begin{document}

\section{Anexo. Acuerdos y pautas de trabajo}
\subsection{Horario de reuniones}

El equipo se reúne con una frecuencia de dos días, aproximadamente. El horario de reuniones queda definido en el calendario de la siguiente manera: 

\begin{itemize}
    \item Lunes: 9:00 a 10:30
    \item Miércoles: 9:00 a 10:30
    \item Viernes: 12:30 a 14:00
    \item Sábado: 16:00 a 17:30
\end{itemize}

\subsection{Acuerdos}

\begin{itemize}
    \item Frecuencia de encuentro: diaria.
    \item Frecuencia de reunión virtual: cada dos días.
    \item Frecuencia de respuesta a mensajes: cada doce horas.
    \item En caso de desaparición, se intentará contactar con la persona ausente. Si no responde, se hablará con el profesorado.
    \item En caso de que alguien deba ausentarse durante más de tres días, deberá avisar presencialmente y dejar la ausencia por escrito en el canal «comunicados» de Discord.
    \item Entorno para intercambiar información: la Wiki de GitHub y los medios de comunicación.
\end{itemize}

Para que una reunión sea efectiva, debe haber mínimo tres de los cuatro miembros presentes. Si no hay asistencia mínima, se intentará cuadrar otro día para que haya mínimo cuatro personas. Además, se intentará que haya un equilibrio entre los miembros del equipo para que no haya sobrecarga de trabajo.\par

El rol de secretario de reunión será otorgado al más joven del equipo, que será el encargado de tomar notas y de redactar el acta de la reunión. Tanto las notas como el acta serán visibles en tiempo real a través de un canal de Discord, y todos los miembros pueden participar. Al terminar cada reunión, el secretario publicará el acta de reunión en la Wiki de GitHub.\par

\subsection{Medios de comunicación}

El equipo emplea distintos medios de comunicación para mantener una relación activa entre los integrantes del equipo y el profesorado de la asignatura. Entre las aplicaciones y plataformas que se usan para tener una comunicación efectiva se encuentran:
\begin{itemize}
    \item Entre el grupo
    \begin{itemize}
        \item WhatsApp
        \item Discord
        \item Outlook/Gmail
    \end{itemize}

    \item Con los profesores
    \begin{itemize}
        \item Outlook
        \item Campus Virtual
    \end{itemize}
    \item GitHub como herramienta de control de versiones y de documentación a través del repositorio y de su Wiki.
\end{itemize}

\end{document}