\documentclass[../main.tex]{subfiles}
\begin{document}
\section{Proceso de desarrollo: SCRUM}

Hemos elegido una metodología ágil para optimizar el desarrollo y la gestión de Mediflix. Dado que somos un equipo pequeño, podemos tener una mayor flexibilidad y adaptabilidad a los cambios que puedan surgir durante el desarrollo del proyecto. Asimismo, nos gusta formentar la comunicación y la participación activa para que todos y cada uno de los miembros de nuestro equipo se sientan comprometidos con Mediflix.\par

Concretamente, el proceso de desarrollo de Mediflix se basa en SCRUM, un enfoque ágil que requiere de reuniones diarias, \textit{sprints} cortos y entregas continuas. Buscamos entregar un producto final de gran calidad, respaldado por una buena gestión de requisitos y de pruebas.\par

Se ha optado por SCRUM en vez de UP (\textit{Unified Process}) para que el equipo aprenda a trabajar en un entorno ágil y pueda aplicar los conocimientos adquiridos en futuros proyectos. Además, queremos fomentar los valores en los que se basa SCRUM (honestidad, apertura, respeto, confianza, etc.).\par

SCRUM se basa en tres roles principales:\par
\begin{itemize}
    \item \textbf{\textit{Product Owner}} (propietario del producto). Experto en el producto y responsable de crear y gestionar los requisitos (\textit{product backlog}).
    \item \textbf{\textit{Scrum Master}} (experto en SCRUM). Responsable de guiar al equipo y de asegurarse de que se siguen los principios y valores de SCRUM.
    \item \textbf{Equipo de desarrollo}: programador senior y programador junior. Su trabajo principal será implementar la arquitectura y los requisitos de Mediflix.
\end{itemize}

Para aumentar la colaboración y el sentimiento de equipo, todos los miembros serán partícipes de todas las reuniones y decisiones del proyecto, así como del desarrollo, de la revisión y la retrospectiva de cada sprint. \par

\subsection{Especificaciones de nuestro SCRUM}

\begin{itemize}
    \item Los \textit{sprints} tendrán una duración de una semana.
    \item Para una comunicación ágil y seguimiento continuo, distinguimos tres tipos de reuniones:
    \begin{itemize}
        \item Encuentros presenciales. Encuentros muy cortos, de lunes a jueves, al salir de clase, para compartir los avances de cada miembro. Duración de menos de quince minutos.
        \item Clases de laboratorio. Durante el horario de laboratorio de las asignaturas de ASEE (miércoles a las 15:30) y GPS (lunes a las 17:30), de una duración de dos horas.
        \item Reuniones virtuales. Reuniones a través de Discord, para revisar y planificar el trabajo. Duración de hora y media.
        \begin{itemize}
            \item Lunes: 9:00 a 10:30
            \item Miércoles: 9:00 a 10:30
            \item Viernes: 12:30 a 14:00
            \item Sábado: 16:00 a 17:30
        \end{itemize}
    \end{itemize}
    \item GitHub como herramienta de control de versiones y de documentación a través del repositorio y de su Wiki.
\end{itemize}

\end{document}